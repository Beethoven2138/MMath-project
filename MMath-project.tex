\documentclass{article}
\usepackage{graphicx} % Required for inserting images
\usepackage[english]{babel}
\usepackage{amssymb}
\usepackage{amsthm}
\usepackage{amsmath}
\usepackage{amsfonts}
\usepackage{tikz}
\usetikzlibrary{matrix}
\usepackage[english]{babel}
\usepackage{mathtools}
\usepackage[a4paper, total={6in, 8in}]{geometry}

\title{MMath project}
\author{Saxon Supple\\Supervisor: Johannes Nordstrom}
\date{January 2024}



\newtheorem{theorem}{Theorem}[section]
\newtheorem{definition}[theorem]{Definition}
\newtheorem{lemma}[theorem]{Lemma}
\newtheorem{proposition}[theorem]{Proposition}
\newtheorem{corollary}[theorem]{Corollary}
\newtheorem{example}[theorem]{Example}
\newtheorem{remark}[theorem]{Remark}

\begin{document}

\maketitle

\section{Introduction}
Lens spaces are quotient spaces of the $3$-sphere by cyclic groups and are the simplest examples of manifolds (topological spaces which locally look like Euclidean space) which are homotopy-equivalent but not homeomorphic. This project will study the conditions under which lens spaces are homotopy-equivalent and homeomorphic. To this end we shall introduce the homology and cohomology groups. Being invariants of homotopy-equivalent spaces, it is expected that they would be insufficient for a classification of manifolds which can be homotopy-equivalent but not homeomorphic. However, the homology and cohomology groups are also insufficient for a homotopy-equivalence classification. The torsion linking form will be introduced for the homotopy-equivalence classification, and Reidemeister torsion for the homeomorphism classification.
\section{Background}
\begin{definition}
Let $(X,\mathcal{T})$ and $(Y,\mathcal{T}')$ be topological spaces. A map $f\colon X\to Y$ is \textbf{continuous} if\\
$f^{-1}(U)\in\mathcal{T}$, $\forall U\in\mathcal{T}'$.
\end{definition}

\begin{definition}
Let $X$ and $Y$ be topological spaces. A continuous map $\phi\colon X\to Y$ is a homeomorphism if it has a continuous inverse $\phi^{-1}\colon Y\to X$.
\end{definition}

\begin{definition}
Let $X$ be a topological space. A \textbf{path} is a continuous map $\gamma$ from $[0,1]$ to $X$. If $\gamma(0)=\gamma(1)$, then $\gamma$ is a \textbf{loop}.
\end{definition}

\begin{definition}
Let $X$ be a topological space and $x_0\in X$. The \textbf{fundamental group} is \[\pi_1(X,x_0)=\{[\gamma]:\gamma\text{ is a loop based at }x_0\}.\]
\end{definition}
This is a group with identity represented by the constant loop, multiplication given by concatenation of loops and $[\alpha]$ having inverse $[\bar \alpha]$, where $\bar \alpha(t)=\alpha(1-t)$.

\begin{theorem}
If $X$ is path-connected and $x_0,y_0\in X$, then $\pi_1(X,x_0)\cong\pi_1(X,y_0)$.
\end{theorem}
We can then refer to the fundamental group merely as $\pi_1(X)$.

\begin{definition}
Two continuous maps $\phi_0,\phi_1\colon X\to Y$ of topological spaces are \textbf{homotopic} if there exists a continuous map $F\colon X\times I\to Y$ such that\[F(x,0)=\phi_0(X), F(x,1)=\phi_1(x),\forall x\in X.\]
In this case, we write: $\phi_0\simeq\phi_1$ and say that $F$ is a \textbf{homotopy from} $\phi_0$ \textbf{to} $\phi_1$.
\end{definition}



\begin{definition}
Let $X$ and $Y$ be topological spaces. If there exist continuous maps $\phi\colon X\to Y$ and $\psi\colon Y\to X$ such that $\phi\circ\psi\simeq\text{id}_Y$ and $\psi\circ\phi\simeq\text{id}_X$, then $X$ and $Y$ are said to be \textbf{homotopy-equivalent}. In this case, we write $X\simeq Y$.
\end{definition}

\begin{theorem}
Homotopy-equivalent path-connected topological spaces have isomorphic fundamental groups.
\end{theorem}

\begin{definition}
Let $G\times E\to E,(g,e)\mapsto g\cdot e$ be a left action of a group $G$ on a topological space $E$.\\
We say that the action is \textbf{continuous} if the left multiplication map $L_g\colon E\to E,e\mapsto g\cdot e$ is continuous for all $g\in G$.\\
We say that the action is a \textbf{covering space action} if it is continuous and every $e\in E$ has an open neighbourhood $e\in S\subset E$ such that $gS\cap S=\emptyset$ for all non-trivial elements $g\in G$.
\end{definition}

\begin{theorem}
Let $G$ be a group acting via a covering space action on a simply connected topological space $E$. Then $\pi_1(E/G,[e])\cong G$ for any $[e]\in E/G$.
\end{theorem}

\section{Singular homology}
\begin{definition}
A \textbf{k-simplex} is the convex hull\[[v_0,...,v_k]:=\{\sum_{i=0}^k\lambda_iv_i:\lambda_i\in [0,1],\sum_{i=0}^k\lambda_i=1\}\]
of $k+1$ linearly independent vectors $v_0,...,v_k\in\mathbb{R}^n$.\\
The \textbf{standard k-simplex} is\[\Delta^k=[e_0,...,e_k]=\{(t_0,...,t_k)\in\mathbb{R}^k:t_i\geq 0, \sum_{i=0}^kt_i=1\}.\]
The vectors $v_0,...,v_k$ define a \textbf{framing} \[F_{v_0,...,v_k}\colon\Delta^k\rightarrow [v_0,...,v_k],(t_i)\mapsto \sum_{i=0}^kt_iv_i.\]
\end{definition}


\noindent For example $\Delta^0$ is a point, $\Delta^1$ a line segment, $\Delta^2$ a triangle and $\Delta^3$ a tetrahedron.

\begin{definition}
The \textbf{faces} of $[v_0,...,v_k]$ are the (k-1)-simplices $[v_0,...,\overset{\wedge}{v_i},...,v_k]$ where $\overset{\wedge}{v_i}$ means omit the entry $v_i$.
\end{definition}

\begin{definition}
Let $X$ be a topological space. A \textbf{singular k-simplex} in X is a continuous map $\sigma\colon\Delta^k\to X$. The \textbf{k-th chain group} of $X$ is the abelian group\[C_k(X):=\{\sum_{i=0}^kn_i\sigma_i:k\in\mathbb{N}_0,n_i\in\mathbb{Z},\sigma_i \text{ is a k-simplex}\}\] where the elements are called \textbf{k-chains}.
\end{definition}
\begin{remark}
$C_k(X)$ is an example of a \textbf{free abelian group}; that is, an abelian group whose elements are of the form $\sum_in_is_i$ where each $s_i$ belongs to some set $S$.
\end{remark}


\begin{definition}
The \textbf{boundary map} $\partial=\partial_k\colon C_k(X)\to C_{k-1}(X)$ is given by\[\partial\sigma=\sum_{i=0}^k(-1)^i\sigma\circ F_{e_0,...,\overset{\wedge}{e_i},...,e_k}\in C_{k-1}(X)\]
on single simplices and extended linearly to all of $C_k(X)$.\\
A k-chain $c\in C_k(X)$ is a \textbf{k-cycle} if $\partial(c)=0$.
\end{definition}
\noindent Intuitively, this definition means that the boundary map decomposes $\sigma$ into a sum of maps to its faces--or rather the images of faces of $\Delta^k$--taking into account orientation.


\begin{lemma}
$\partial\circ\partial\colon C_k(X)\to C_{k-2}(x)$ is zero.
\end{lemma}
\begin{proof}
Let $\sigma\in C_k(X)$.
\begin{align*}
\partial\circ\partial (\sigma)&=\partial(\sum_{i=0}^k(-1)^i\sigma\circ F_{[e_0,...,\overset{\wedge}{e_i},...,e_k]})\\
&=\sum_{i=0}^k(-1)^i\partial\circ\sigma\circ F_{[e_0,...,\overset{\wedge}{e_i},...,e_k]} \\&\text{by linearity}\\
&=\sum_{i=0}^k(\sum_{j<i}(-1)^{i+j}\sigma\circ F_{[e_0,...,\overset{\wedge}{e_j},...,\overset{\wedge}{e_i},...,e_k]} + \sum_{i<j}(-1)^{i+j-1}\sigma\circ F_{[e_0,...,\overset{\wedge}{e_i},...,\overset{\wedge}{e_j},...,e_k]}) \\&\text{(since }e_j \text{ gets shifted forward one spot when }e_i\text{ is removed first)}\\
&=\sum_{i=0}^k(\sum_{j<i}(-1)^{i+j}\sigma\circ F_{[e_0,...,\overset{\wedge}{e_j},...,\overset{\wedge}{e_i},...,e_k]} + \sum_{j<i}(-1)^{i+j-1}\sigma\circ F_{[e_0,...,\overset{\wedge}{e_j},...,\overset{\wedge}{e_i},...,e_k]})
\\&\text{(interchanging }j\text{ with }i\text{ in the second term)}\\
&=\sum_{i=0}^k(\sum_{j<i}((-1)^{i+j}+(-1)^{1+j-1})\sigma\circ F_{[e_0,...,\overset{\wedge}{e_j},...,\overset{\wedge}{e_i},...,e_k]})=0 \qedhere
\end{align*}
\end{proof}

\noindent A simple example of this fact is that the boundary of a ball is a sphere, which in turn has no boundary.\\
It also implies that $\text{im}(\partial:C_{k+1}(X)\rightarrow C_k(X))\in\text{ker}(\partial:C_{k}(X)\rightarrow C_{k-1}(X))$,
allowing the following to be well-defined.
\begin{definition}
The \textbf{degree k singular homology} of $X$ is the abelian group
\[H_k(X)=\frac{\text{ker}(\partial:C_{k}(x)\rightarrow C_{k-1}(X))}{\text{im}(\partial:C_{k+1}(x)\rightarrow C_k(X))}\]
\end{definition}

\noindent The utility of the homology groups are that homotopy-equivalent spaces have isomorphic homology groups, thereby giving a way to prove that two topological spaces aren't homotopy-equivalent. To prove this, we must first construct a functor sending a continuous map between topological spaces to a homomorphism between their homology groups.\\

\noindent To give a more general result we shall allow the chain groups to have coefficients in any abelian group $G$ as opposed to merely $\mathbb{Z}$.
\begin{definition}
The k-th singular chain group with coefficients in an abelian group $G$ is \[C_k(X;G):=\{\sum_{i=1}^kn_i\sigma_i:k\in\mathbb{N}_0,n_i\in G,\sigma_i \text{ is a k-simplex}\}.\] The boundary map $\partial\colon C_k(X;G)\to C_{k-1}(X;G)$ and k-th homology group with coefficients in $G$, $H_k(X;G)$, are then defined as above while replacing $\mathbb{Z}$ with $G$.
\end{definition}

\noindent Let $f\colon X\to Y$ be a continuous map between topological spaces. Given a k--simplex $\sigma\colon\Delta^k\to X$ in $X$, $f\circ\sigma$ is a k--simplex in $Y$. This gives rise to a homomorphism \[f_\#\colon C_k(X;G)\to C_k(Y;G):\sum_{i=0}^kn_i\sigma_i\mapsto\sum_{i=0}^kn_if\circ\sigma_i.\]
\begin{definition}
This induces a \textbf{push-forward homomorphism} $f_*\colon H_k(X;G)\to H_k(Y;G)$ sending $[c]$ to $[f_\#(c)]$.
\end{definition}
\noindent To show that this is well-defined, note that $\partial f_\#=f_\#\partial$ and so $f_\#$ maps closed chains to closed chains. Furthermore suppose $[a-b]=[0]$. Then there exists a $\sigma\in C_{k+1}(X;G)$ such that \[a-b=\partial\sigma\implies f_\#(a-b)=f_\#\partial\sigma=\partial f_\#\sigma\implies [f_\#(a)-f_\#(b)]=[0]=f_*([a])-f_*([b])\] as required.\\

\noindent The condition $\partial f_\#=f_\#\partial$ makes $f_\#$ a \textbf{chain map}. This argument applies more generally to show that all chain maps induce a well-defined map on homology groups.

\begin{proposition}
The push-forward is functorial.
\end{proposition}
\begin{proof}
Let $X$, $Y$ and $Z$ be topological spaces.
Clearly $\text{id}_{X*}=\text{id}_{H_k(X;G)}$.\\
Now let $f\colon Y\to Z$ and $g\colon X\to X$ be continuous maps. Given $[c]=[\sum_{i=0}^kn_i\sigma_i]\in H_k(X;G)$, \[(f\circ g)_*([c])=[\sum_{i=0}^kn_if\circ g(\sigma_i)]=f_*([\sum_{i=0}^kn_ig(\sigma_i)])=f_*\circ g_*([c])\] so $(f\circ g)_*=f_*\circ g_*$.
\end{proof}

\noindent This then implies the standard results from category theory such as the push-forward sending commutative diagrams to commutative diagrams and homeomorphisms to group isomorphisms.

\begin{lemma}
If $f,g\colon X\to Y$ are homotopic continuous maps between topological spaces then $f_*=g_*\colon H_k(X)\to H_k(Y)$
\end{lemma}
\begin{proof}
Let $F\colon X\times I\to Y$ be a homotopy from $f$ to $g$. Given a singular n--simplex $\sigma\colon\Delta^n\to X$ we can define $\sigma\times\text{id}\colon\Delta^n\times I\to X\times I$ and $F_\#(\sigma\times\text{id})\colon\Delta^n\times I\to Y$. We now decompose $\Delta^n\times I$ as the union of $n+1$-simplices. To do this write $\Delta^n\times I$ as $[a_0,...,a_n,b_0,...,b_n]$ where $a_i=(e_i,0)$ and $b_i=(e_i,1)$. Then $\Delta^n\times I$ is the union of the $n+1$-simplices $\alpha_i=[a_0,...,a_i,b_i,...,b_n]$. Now define the "prism map" $P\colon C_n(X)\to C_{n+1}(Y)$ by $P(\sigma)=\sum_i(-1)^iF\circ(\sigma\times\text{id})|_{\alpha_i}$. 
\begin{align*}
\partial P(\sigma)&=\sum_i(-1)^i[\sum_{j\leq i}(-1)^jF\circ(\sigma\times\text{id})|_{[a_0,...,\overset{\wedge}{a_j},...,a_i,b_i,...,b_n]}\\&+\sum_{j> i}(-1)^{(j+1)}F\circ(\sigma\times\text{id})|_{[a_0,...,a_i,b_i,...,\overset{\wedge}{b_j},...,b_n]}].
\end{align*}
The sum of terms with $i=j$ is 
\begin{multline*}
F\circ(\sigma\times\text{id})|_{[\overset{\wedge}{a_0},b_0,...,b_n]}-F\circ(\sigma\times\text{id})|_{[a_0,\overset{\wedge}{b_0},b_1,...,b_n]}+F\circ(\sigma\times\text{id})|_{[a_0,\overset{\wedge}{a_1},b_1,...,b_n]}\\-F\circ(\sigma\times\text{id})|_{[a_0,a_1,\overset{\wedge}{b_1},...,b_n]}+...+F\circ(\sigma\times\text{id})|_{[a_0,...,\overset{\wedge}{a_n},b_n]}-F\circ(\sigma\times\text{id})|_{[a_0,...,a_n,\overset{\wedge}{b_n}]}\\=F\circ(\sigma\times\text{id})|_{[\overset{\wedge}{a_0},b_0,...,b_n]}-F\circ(\sigma\times\text{id})|_{[a_0,...,a_n,\overset{\wedge}{b_n}]}=F\circ(\sigma\times\text{id})|_{[b_0,...,b_n]}-F\circ(\sigma\times\text{id})|_{[a_0,...,a_n]}
\end{multline*}as a telescoping sum which is then $=g\circ\sigma-f\circ\sigma=(g_\#-f_\#)(\sigma)$. The terms with $i\neq j$ can be identified with $-P(\partial\sigma)$ so that $\partial P\sigma=-P(\partial\sigma)+g_\#(\sigma)-f_\#(\sigma)\implies\partial P + P\partial=g_\#-f_\#$. $\partial(\partial P + P\partial)=\partial P\partial=(\partial P + P\partial)\partial$ so $(\partial P + P\partial)$ is a chain map inducing a homomorphism on homology groups. Furthermore given a cycle $\sigma$ we have $(\partial P + P\partial)(\sigma)=\partial P(\sigma) + P\partial(\sigma)=\partial P(\sigma)$ so $\partial P(\sigma)$ sends cycles to boundaries thereby inducing the zero map on homology groups. Thus $g_*-f_*=0$ or $f_*=g_*$.
\end{proof}


\noindent We are now ready to prove that homology is a topological invariant.
\begin{proposition}
If $X$ and $Y$ are homotopy-equivalent then $H_k(X;G)\cong H_k(Y;G)$
\end{proposition}
\begin{proof}
We have continuous maps $f\colon X\to Y$ and $g\colon Y\to X$ such that $g\circ f\simeq\text{id}_X$ and $f\circ g\simeq\text{id}_Y$ giving $g_*\circ f_*=\text{id}_{H_k(X;G)}$ and $f_*\circ g_*=\text{id}_{H_k(Y;G)}$. Thus $f_*$ has a left and right inverse so is a bijection so a group isomorphism.
\end{proof}

\begin{lemma}
If $X$ is path-connected then $H_0(X)=\mathbb{Z}$.
\end{lemma}
\begin{proof}
A singular 0-simplex is a map from a point to a point $x\in X$. let $c=\sum n_i\sigma_i$ be a 0-chain. $\partial c=0$ so $\text{ker}(\partial:C_{0}(x)\rightarrow C_{-1}(X))=C_0(X)$ giving $H_0(X)=\frac{C_0(X)}{\partial C_1(X)}$. Let $x,y\in X$ and let $\sigma_x$ and $\sigma_y$ be their corresponding simplicial 0-simplices. $X$ is path-connected so there exists a simplicial 1-simplex $\sigma$ with $\partial\sigma=\sigma_x-\sigma_y$. Define $f\colon C_0(X)\to\mathbb{Z}:\sum n_ix_i\mapsto\sum n_i$. Clearly $\partial C_1(X)\in \text{ker}f$. Let $z=\sum_{i=0}^ka_i\sigma_i\in\text{ker}f$. Express $a_i\sigma_i$ as $\sum_{j=1}^{|a_i|}a_{ij}\sigma_i$ where $a_{ij}$ is either all $1$ or all $-1$. Thus $z=\sum_{j=1}^rb_j\tau_j$ where $\tau_j\in\{\sigma_0,...,\sigma_k\}$ and $b_j=\pm1$. $|\{j:b_j=1\}|=|\{j:b_j=-1\}|$ so $r$ is even and we can express $z$ as a sum of terms of the form $(\sigma_\alpha - \sigma_\beta)\in\partial C_1(X)$, giving $\text{ker}f\in\partial C_1(X)$. Thus $\text{ker}f\in\partial C_1(X)$. $f$ is surjective so the result follows by the first isomorphism theorem.
\end{proof}

\begin{lemma}
$H_k(\{\text{pt}\})=\begin{cases}
       \mathbb{Z} &\quad\text{if }k=0 \\
       0 &\quad\text{otherwise} \\ 
     \end{cases}$
\end{lemma}
\begin{proof}
There is only one k-simplex $\sigma_k\colon\Delta^k\rightarrow \{\text{pt}\}$, the constant map. Thus $C_k(\{\text{pt}\})=\{n\sigma_k:n\in\mathbb{Z}\}=\mathbb{Z}$.
Also $\partial_k(n\sigma_k)=\sum_{i=0}^k(-1)^i n\sigma_{k-1}$ which is zero if $k$ is odd and $n\sigma_{k-1}$ if $k$ is even. Thus the induced map on the coefficients of the boundary is $0$ if $k$ is odd and id if $k$ is even. The Homology groups will then be $\frac{\mathbb{Z}}{\mathbb{Z}}=0$ or $\frac{\{0\}}{\{0\}}=0$, except for $H_0$ which is $\mathbb{Z}$ since $\{\text{pt}\}$ is path-connected.
\end{proof}


\begin{definition}
The \textbf{commutator subgroup} of a group $G$, denoted $[G,G]$, is the subgroup generated by all elements, called $commutators$, of the form $ab(ba)^{-1}$ for any $a$ and $b$ in $G$.
\end{definition}

\begin{lemma}
$[G,G]$ is a normal subgroup.
\end{lemma}
\begin{proof}
Let $g\in G,aba^{-1}b^{-1}\in [G,G]$.\\
\begin{align*}
g^{-1}(aba^{-1}b^{-1})g&=g^{-1}agg^{-1}bgg^{-1}a^{-1}gg^{-1}b^{-1}g\\&=(g^{-1}ag)(g^{-1}bg)(g^{-1}a^{-1}g)(g^{-1}b^{-1}g)\\&=(g^{-1}ag)(g^{-1}bg)(g^{-1}ag)^{-1}(g^{-1}bg)^{-1}\in[G,G].\qedhere
\end{align*}
\end{proof}
\begin{definition}
The \textbf{abelianization} $G^{\text{ab}}$ of a group $G$ is $G/[G,G]$.
\end{definition}
\noindent As the name suggests, $G^{\text{ab}}$ is abelian since we quotiented out by the relation that $ab=ba$ for all $a,b\in G$. If $G$ is already abelian then $[G,G]$ is trivial so $G^{\text{ab}}=G$.

\begin{lemma}
Given groups $G$ and $H$, where $H$ is abelian, and a group homomorphism $h\colon G\to H$, the induced homomorphism $h'\colon G^{\text{ab}}\to H:[x]\mapsto h(x)$ is well-defined.
\end{lemma}
\begin{proof}
Let $[x]=[y]\in G^{\text{ab}}$ so that $x=y\prod_ia_ib_ia_i^{-1}b_i^{-1}$ for $a_i,b_i\in G$.
Then \[h'([x])=h(y\prod_ia_ib_ia_i^{-1}b_i^{-1})=h(y)\prod_ih(a_i)h(b_i)h(a_i)^{-1}h(b_i)^{-1}=h(y)=h'([y])\] since $H$ is abelian.
\end{proof}

\begin{lemma}
Let $G$ be a group. If $w$ is a word in $G$ such that for any $g\in G$ the exponents of $g$ in $w$ sum to zero then $w\in[G,G]$ 
\end{lemma}
\begin{proof}
Consider the quotient map $\pi\colon G\to G/[G,G]:g\mapsto[g]$.
$G/[G,G]$ is abelian so the letters in $[w]$ can be reordered such that $[w]=\sum n_i[a_i]=0$ implying $w\in\text{ker }\pi=[G,G]$. 
\end{proof}

\begin{theorem}
(Hurewicz). If $X$ is path-connected then $H_1(X)\cong\pi_1(X)^{\text{ab}}$.
\end{theorem}
\begin{proof}
First note that an element of $\pi_1(X,x_0)$ is of the form $[\gamma]$ for $\gamma\colon [0,1]\to X$ a loop based at $x_0$. $[0,1]=\Delta^1$ so $\gamma$ is a 1-simplex and $\partial(\gamma)=x_0-x_0$ making $\gamma$ closed.
We can then define $h\colon\pi_1(X,x_0)\to H_1(X)$ sending an element $[\gamma]\in\pi_1(X,x_0)$ to $[c_\gamma]\in H_1(X)$ where $c_\gamma$ is $\gamma$ as an element of $C_1(X)$. To show that $h$ is well-defined let $[a]=[b]\in\pi_1(X,x_0)$ and let $F\colon [0,1]^2\to X$ be a based-homotopy from $a$ to $b$. Dividing $[0,1]^2$ into two triangles and parametrising each by $\Delta^2$ allows F to be expressed as $c_F\in C_2(X)$ with $\partial c_F=c_a-c_b\implies [c_a]=[c_b]$ as required.

\noindent \textbf{Next we show that $h$ is a homomorphism.}\\
Let $[f],[g]\in\pi_1(X,x_0)$ and Define a singular 2-simplex $\sigma\colon\Delta^2\to X$ where $\sigma_{|[e_0,e_1]}=f$, $\sigma_{|[e_1,e_2]}=g$ and $\sigma_{|[e_0,e_2]}=f\cdot g$. Then $\partial(\sigma)=g-f\cdot g+f$ so $h([f][g])=h([f])+h([g])$ making $h$ a homomorphism.

\noindent \textbf{We now show that $h$ is surjective.}\\
For the sake of lucidity we will write $[\gamma]$ for $h([\gamma])$ instead of $[c_\gamma]$\\
Let $[\lambda]\in H_1(X)$ where $\lambda=\sum_{i=0}^k n_i\sigma_i$.
$\partial (\lambda)=0=\sum_{i=0}^k n_i(\sigma_i(1)-\sigma_i(0))$.\\
Let $\Lambda=\{\sigma_i(0)\}\cup\{\sigma_i(1)\}$ be the collection of end-points of the $\sigma_i$s.
For each $p\in\Lambda$ pick a path $\beta_p\colon [0,1]\to X$ from $x_0$ to $p$ (which is possible because $X$ is path-connected). Then \[\sum_{i=0}^k n_i(\sigma_i(1)-\sigma_i(0))=\sum m_pp=0\] where $m_p$ is the sum of the coefficients of $p$ in the sum. This implies that $m_p=0 \forall p$. Now consider the loop $\eta_i=\beta_{\sigma_i(0)}\cdot\sigma_i\cdot\beta_{\sigma_i(1)}^{-1}$ based at $x_0$. Then 
\begin{align*}
h([\eta_0^{n_0}\cdot\eta_1^{n_1}...\cdot\eta_k^{n_k}])&=[n_0\eta_0+...+n_k\eta_k]\\&=[\sum_{i=0}^kn_i(\beta_{\sigma_i(0)}+\sigma_i-\beta_{\sigma_i(1)})]\\&=[\sum_{i=0}^kn_i\sigma_i+\sum_{i=0}^kn_i(\beta_{\sigma_i(0)}-\beta_{\sigma_i(1)})]\\&=[\sum_{i=0}^kn_i\sigma_i-\sum m_p\beta_p]\\&=[\sum_{i=0}^kn_i\sigma_i]=[\lambda]
\end{align*}since $m_p=0$.\\
\noindent\textbf{We now find the kernel.}\\
Let $[\gamma]\in\pi_1(X,x_0)$ such that $h([\gamma])=[0]$.
Then $\gamma$ is exact so there exists a finite sum of singular 2-simplices $\sum_{i=0}^kn_i\sigma_i$ such that \[\gamma=\partial(\sum_{i=0}^kn_i\sigma_i)=\sum_{i=0}^kn_i\partial(\sigma_i)=\sum_{i=0}^kn_i(\lambda_i+\mu_i+\nu_i)\] Let $\Lambda=\{\lambda_i\}\cup\{\mu_i\}\cup\{\nu_i\}$ and for each end-point $p$ of a singular 1-simplex in $\Lambda$ let $\beta_p$ be a path from $x_0$ to $p$. For each $\theta\in\Lambda$ let $m_\theta$ be the sum of coefficients of $\theta$ in $\sum_{i=0}^kn_i(\lambda_i+\mu_i+\nu_i)$. This sum is equal to $\gamma$ thus $m_\theta$ is $1$ if $\theta=\gamma$ and $0$ otherwise. For each singular 2-simplex $\sigma_i$ the loops $\beta_{\lambda_i(0)}\cdot\lambda_i\cdot\beta_{\lambda_i(1)}^{-1}$, $\beta_{\mu_i(0)}\cdot\mu_i\cdot\beta_{\mu_i(1)}^{-1}$ and $\beta_{\nu_i(0)}\cdot\nu_i\cdot\beta_{\nu_i(1)}^{-1}$ have null-homotopic concatenations. Let $\eta_i=\beta_{\partial\sigma_i(0)}\cdot\partial\sigma_i\cdot\beta_{\partial\sigma_i(1)}^{-1}$. Then $h([\eta_0^{n_0}\cdot...\cdot\eta_k^{n_k}\cdot\gamma^{-1}])=h([\gamma])h([\gamma^{-1}])=[\gamma^{-1}]$. Now write each $[\eta_i]$ as $[\beta_{\lambda_i(0)}\cdot\lambda_i\cdot\beta_{\lambda_i(1)}^{-1}][\beta_{\mu_i(0)}\cdot\mu_i\cdot\beta_{\mu_i(1)}^{-1}][\beta_{\nu_i(0)}\cdot\nu_i\cdot\beta_{\nu_i(1)}^{-1}]$. Then for each $\theta\in\Lambda$ we have that the sum of exponents of $[\beta_{\theta(0)}\cdot\theta\cdot\beta_{\theta(1)}^{-1}]$ is $m_\theta$ The exponents of each $[\beta_{\theta(0)}\cdot\theta\cdot\beta_{\theta(1)}^{-1}]$ in $[\eta_0^{n_0}\cdot...\cdot\eta_k^{n_k}\cdot\gamma^{-1}]$ is then 0 so by the above lemma $[\gamma^{-1}]\in[\pi_1(X,x_0),\pi_1(X,x_0)]$ meaning $[\gamma]$ is as well. Thus $\text{ker }h\subseteq [\pi_1(X,x_0),\pi_1(X,x_0)]$. We also have that $[\pi_1(X,x_0),\pi_1(X,x_0)]\subseteq \text{ker }h$ since $H_1(X)$ is abelian so $\text{ker }h=[\pi_1(X,x_0),\pi_1(X,x_0)]$. The result then follows by the first isomorphism theorem.
\end{proof}

\begin{corollary}
\begin{align*}
H_1(S^n)&=\begin{cases}
       \mathbb{Z} &\quad\text{if }n=1 \\
       0 &\quad\text{otherwise} \\ 
     \end{cases}\\
H_1(T^2)&=\mathbb{Z}^2\\
H_1(\mathbb{R}P^n)&=\begin{cases}
       \mathbb{Z} &\quad\text{if }n=1 \\
       \mathbb{Z}_2 &\quad\text{otherwise} \\ 
     \end{cases}\\
H_1(K)&=\mathbb{Z}\bigoplus\mathbb{Z}_2
\end{align*}
\end{corollary}

\section{Cohomology}

\begin{definition}
Let $R$ be a commutative ring with a unit. An \textbf{R-module} $M$ is an abelian group with a map $f\colon R\times M\to M$ such that if $f(r,m)$ is denoted by $rm$ then $(r_1r_2)m=r_1(r_2m)$ and $1m=m$. We usually abbreviate R-modules to modules.
\end{definition}
\begin{remark}
More specifically we defined a \textbf{left R-module}. If instead multiplication were defined to satisfy $r_1(r_2m)=(r_2r_1)m$ then $M$ would be a \textbf{right R-module}.
\end{remark}

\begin{definition}
Let $M$ and $N$ be R-modules. An \textbf{R-module homomorphism} $f\colon M\to N$ is a homomorphism of abelian groups such that $f(rm)=rf(m)$ 
\end{definition}

\begin{example}
An abelian group $G$ is a $\mathbb{Z}$-module given by $ng=$$\begin{cases}
       \sum_{i=1}^ng &\quad\text{if }n>0 \\
       0 &\quad\text{if }n=0 \\
       -\sum_{i=1}^{-n}g &\quad\text{if }n<0 \\
     \end{cases}$\\
\end{example}

\begin{definition}
Let $A$ and $B$ be R-modules. Their \textbf{direct sum} $A\bigoplus B$ is the module $A\times B$ under component-wise operations i.e. $(a_1,b_1)+(a_2,b_2)=(a_1+a_2,b_1+b_2)$ and $r(a,b)=(ra,rb)$. This can be extended to any finite collection of R-modules.
\end{definition}

\begin{definition}
A bilinear map $h\colon M\times N\to P$ is a map such that $h(m_1+m_2,n)=h(m_1,n)+h(m_2,n)$, $h(m,n_1+n_2)=h(m,n_1)+h(m,n_2)$ and $h(rm,n)=h(m,rn)$ for $r\in R$.
\end{definition}

\begin{definition}
Given a ring $R$ and set $S$, we define the module $R^S$ to be functions from $S$ to $R$, with addition and multiplication defined piecewise.
\end{definition}


\begin{definition}
Given a module M, we say that $H\subseteq M$ is a \textbf{submodule} if $rh\in H\forall r\in R,h\in H$ and $H$ is itself a module after inheriting the addition and multiplication maps of $M$.
\end{definition}

\begin{definition}
Given a module $M$ and submodule $N$, the \textbf{quotient R-module of M by N} is the quotient group $M/N$ with multiplication given by $r(a+N)=ra+N\forall r\in R,a+N\in M/N$.
\end{definition}

\begin{definition}
Let $M$ and $N$ be modules. Let $I$ denote the submodule of $M\times N$ generated by the set of elements of the form $(\lambda m_1+m_2,n)-\lambda(m_1,n)-(m_2,n)$, $(m,\lambda n_1+n_2)-\lambda(m,n_1)-(m,n_2)$ for $m,m_1,m_2\in M$, $n,n_1,n_2\in N$ and $\lambda\in R$. Then the \textbf{tensor product} of $M$ and $N$ is $M\bigotimes_R N=R^{M\times N}/I$ whose elements are represented as $m\otimes n$ for $m\in M$, $n\in N$.
\end{definition}

\noindent This gives rise to the relations\\ $(m_1+m_2)\otimes n=m_1\otimes n+m_2\otimes n$,\\
$m\otimes(n_1+n_2)=m\otimes n_1+m\otimes n_2$ and\\
$(rm)\otimes n=m\otimes (rn)=r(m\otimes n)$ for $r\in R$.

\begin{example}
If $p$ and $q$ are coprime then $\mathbb{Z}_p\bigotimes_\mathbb{Z}\mathbb{Z}_q=0$ since there exist $m,n\in\mathbb{Z}$ such that $mp+nq=1$ (Bezout's lemma) so $a\otimes b=(mp+nq)a\otimes b=(mpa)\otimes b + a\otimes(nqb)=0\forall a\otimes b\in\mathbb{Z}_p\bigotimes_\mathbb{Z}\mathbb{Z}_q$.
\end{example}
\begin{example}
$C_k(X;G)=C_k(X)\bigotimes_\mathbb{Z}G$.
\end{example}

\begin{example}
$\mathbb{Z}_m\bigotimes\mathbb{Z}_n=\mathbb{Z}_{gcd(m,n)}$.
\end{example}
\begin{proof}
We have the first isomorphism for modules that given a module homomorphism $f:M\to N$, $M/\text{ker }f\cong\text{im }f$ (the proof is analogous to the cases for groups, rings and vector spaces).
So, define a module homomorphism $g\colon\mathbb{Z}\to\mathbb{Z}_m\bigotimes\mathbb{Z}_n:z\mapsto z(1\otimes 1)$. $g$ is surjective so $\text{im }g=\mathbb{Z}_m\bigotimes\mathbb{Z}_n$. Clearly $\langle\text{gcd}(m,n)\rangle\subseteq\text{ker }g$. Now let $x\notin \langle\text{gcd}(m,n)\rangle$. Then $x=s\text{gcd}(m,n)+k$ with $0<k<\text{gcd}(m,n)$ by division with remainder. $g(x)=s\text{gcd}(m,n)(1\otimes 1)+g(k)=k(1\otimes 1)\neq 0\implies x\notin\text{ker }g$ so $x\in\text{ker }g\implies x\in\langle\text{gcd}(m,n)\rangle$ giving $\langle\text{gcd}(m,n)\rangle=\text{ker }g$.
\end{proof}


\begin{definition}
Let M and N be modules over a ring R. The set of all module homomorphisms from M to N is denoted by $\text{Hom}_R(M,N)$. It is an abelian group under pointwise addition and a module if R is commutative.
\end{definition}

\begin{definition}
Let $G$ be an abelian group. The \textbf{torsion subgroup} $T(G)$ is the subgroup consisting of all elements of $G$ with finite order. For a prime number p, the \textbf{p-torsion} subgroup $T_p(G)$ is the subgroup consisting of all elements with order a power of p. The \textbf{free part} of $G$, denoted $F(G)$, is the subgroup comprising all elements of infinite order, as well as the identity.
\end{definition}

\noindent We then have that $G=T(G)\bigoplus F(G)$ for any abelian group $G$.


\begin{definition}
Let $X$ be a topological space and let $G$ be an abelian group. The \textbf{kth cochain group with coefficients in G} is $C^k(X;G)=Hom(C_k(X),G)$. 
\end{definition}

\begin{definition}
We define $d\colon C^k(X;G)\to C^{k+1}(X;G)$ to be the dual of $\partial$ in the sense that if $\phi\in C^k(X;G)$ and $c\in C_{k+1}(X)$ then $(d\phi)(c)=\phi(\partial c)\in G$
\end{definition}

\noindent In particular, $(d\circ d\phi)(c)=(d\phi)(\partial c)=\phi(\partial\circ\partial c)=\phi(0)=0$ so $d^2=0$.

\begin{definition}
$c\in C_k(X)$ is a cycle if $\partial c=0$ and a boundary if $c\in\text{im }\partial$.
$\phi\in C^k(X;G)$ is a cocycle if $d\phi=0$ and a coboundary if $\phi\in\text{im }d$.
\end{definition}

\begin{definition}
The \textbf{kth cohomology group with coefficients in $G$} is \[H^k(X;G)=\frac{\text{ker}(d\colon C^k(X;G)\to C^{k+1}(X;G))}{\text{im}(d\colon C^{k-1}(X;G)\to C^k(X;G))}\] When G is $\mathbb{Z}$ we abbreviate it to $H^k(X)$ and call it simply the \textbf{kth cohomology group}.
\end{definition}

\begin{definition}
Let $f\colon X\to Y$ be a continuous map between topological spaces and $G$ be an abelian group. We have a homomorphism $f^\#\colon C^k(Y;G)\to C^k(X;G):A\mapsto f^\# A$ given by $f^\# A(c)=A(f_\#(c))$. This in turn induces the \textbf{pullback homomorphism} \[f^*\colon H^k(Y;G)\to H^k(X;G):[A]\mapsto [f^\# A].\]
\end{definition}
\noindent $df^\# A(c)=f^\# A(\partial c)=f^\# dA(c)$ so $df^\#=f^\# d$. Thus $f^\#$ is a chain map so $f^*$ is well-defined.

\begin{lemma}
There is a well-defined pairing $H^k(X;G)\times H_k(X)\to G$ given by $([\phi],[c])=\phi(c)$.
\end{lemma}
\begin{proof}
Let $[\phi]=[\psi]$ so that $\phi-\psi=d\rho$ for some $\rho\in C^{k-1}(X;G)$. Then $(\phi-\psi)(c)=d\rho(c)=\rho(\partial c)=0$ since $c\in\text{ker }\partial$. Thus $\phi(c)=\psi(c)$.
Now let $[a]=[b]$ so that $a=b+\partial e$ for some $e\in C_{k+1}(X)$. $\phi(a)=\phi(b+\partial e)=\phi(b)+\phi(\partial e)=\phi(b)+d\phi(e)=\phi(b)$ since $\phi\in\text{ker }d$.
\end{proof}

\noindent This then give that $f^*$ is dual to $f_*$ in the sense that $f^*[A]([c])=[A](f_*([c]))$ for $[A]\in H^k(Y;G)$ and $[c]\in H_k(X)$.

\begin{proposition}
Pullback is covariantly functorial.
\end{proposition}
\begin{proof}
Let $X$, $Y$ and $Z$ be topological spaces and let $G$ be an abelian group. 
Clearly $\text{id}_X^*=\text{id}_{H^k(X;G)}$.
Now let $f\colon X\to Y$ and $g\colon Y\to Z$ be continuous maps. \[(g\circ f)^*([A])=[(g\circ f)^\#(A)]=[f^\#\circ g^\#(A)]=f^*([g^\#(A)])=f^*\circ g^*([A])\forall [A]\in H^k(Z;G).\] Thus $(g\circ f)^*=f^*\circ g^*$.
\end{proof}

\begin{proposition}
Let $X$ and $Y$ be homotopy-equivalent topological spaces. Then \[H^k(X;G)\cong H^k(Y;G).\]
\end{proposition}
\begin{proof}
We have continuous maps $f\colon X\to Y$ and $g\colon Y\to X$ such that $g\circ f\simeq\text{id}_X$ and $f\circ g\simeq\text{id}_Y$. $(g\circ f)_*=\text{id}_{H_k(X)}$ so $f^*\circ g^*=(g\circ f)^*=\text{id}_{H^k(X;G)}$ by duality. Similarly $g^*\circ f^*=\text{id}_{H^k(Y;G)}$. Thus $f^*$ is a bijective module homomorphism so an isomorphism.
\end{proof}


\section{Cup product and ring structure of cohomology}
\begin{definition}
Let $R$ be a ring and $X$ be a topological space. The \textbf{cup product} is the product on cochains\[C^i(X;R)\times C^j(X;R)\to C^{i+j}(X;R):(\psi,\phi)\mapsto\psi\cup\phi\] where \[(\psi\cup\phi)\sigma=\psi(\sigma\circ F_{[e_0,...,e_i]})\phi(\sigma\circ F_{[e_i,...,e_{i+j}]})\] for a singular $i+j$-simplex $\sigma\colon\Delta^{i+j}\to X$ and is extended linearly to $C_{i+j}(X)$.
\end{definition}

\begin{lemma}
$d(\psi\cup\phi)=(d\psi)\cup\phi+(-1)^i\psi\cup d\phi$.
\end{lemma}
\begin{proof}
Let $\sigma\colon\Delta^{i+j+1}\to X$ be a singular simplex.

\begin{align*}
d(\psi\cup\phi)(\sigma)&=\psi\cup\phi(\partial\sigma)\\&=\psi\cup\phi(\sum_{t=0}^{j+k+1}(-1)^t\sigma\circ F_{[e_0,...,\overset{\wedge}{e_t},...,e_{i+j+1}]})\\&=\sum_{t=0}^{i+j+1}(-1)^t\psi\cup\phi(\sigma\circ F_{[e_0,...,\overset{\wedge}{e_t},...,e_{i+j+1}]})\\&=\sum_{t=0}^{i}(-1)^t\psi\cup\phi(\sigma\circ F_{[e_0,...,\overset{\wedge}{e_t},...,e_{i+j+1}]})+\sum_{t=i+1}^{i+j+1}(-1)^t\psi\cup\phi(\sigma\circ F_{[e_0,...,\overset{\wedge}{e_t},...,e_{i+j+1}]})\\&=\sum_{t=0}^i(-1)^t\psi(\sigma\circ F_{[e_0,...,\overset{\wedge}{e_t},...,e_{i+1}]})\phi(\sigma\circ F_{[e_{i+1},...,e_{i+j+1}]})\\&+\sum_{t=i+1}^{i+j+1}(-1)^t\psi(\sigma\circ F_{[e_0,...,e_{i}]})\phi(\sigma\circ F_{[e_{i},...,\overset{\wedge}{e_t},...,e_{i+j+1}]})\\
&=\sum_{t=0}^{i+1}(-1)^t\psi(\sigma\circ F_{[e_0,...,\overset{\wedge}{e_t},...,e_{i+1}]})\phi(\sigma\circ F_{[e_{i+1},...,e_{i+j+1}]})\\&+\sum_{t=i}^{i+j+1}(-1)^t\psi(\sigma\circ F_{[e_0,...,e_{i}]})\phi(\sigma\circ F_{[e_{i},...,\overset{\wedge}{e_t},...,e_{i+j+1}]}).
\end{align*}

\noindent We have \[(d\psi\cup\phi)(\sigma)=\sum_{t=0}^{i+1}(-1)^t\psi(\sigma\circ F_{[e_0,...,\overset{\wedge}{e_t},...,e_{i+1}]})\phi(\sigma\circ F_{[e_{i+1},...,e_{i+j+1}]})\]
and \[(-1)^i(\psi\cup d\phi)(\sigma)=\sum_{t=i}^{i+j+1}(-1)^t\psi(\sigma\circ F_{[e_0,...,e_i]})\phi(\sigma\circ F_{[e_i,...,\overset{\wedge}{e_t},...,e_{i+j+1}]})\] When these expressions are added, the last term of the first sum cancels with the first term of the second sum, leaving $d(\psi\cup\phi)(\sigma)$.
\end{proof}

\begin{proposition}
There is a well-defined pairing $H^i(X;R)\times H^j(X;R)\to H^{i+j}(X;R)$ given by $([\psi],[\phi])\mapsto [\psi]\cup [\phi]=[\psi\cup\phi]$.
\end{proposition}
\begin{proof}
$d(\psi\cup\phi)=0\cup\phi+(-1)^i\psi\cup 0=0$ so $\psi\cup\phi \in \text{ker }d$.
Now let $\psi=df$. Then \[d(f\cup\phi)=(df)\cup\phi\pm f\cup d\phi=\psi\cup\phi+f\cup0=\psi\cup\phi.\] Similarly if $\phi=dg$ then $d((-1)^i\psi\cup g)=\psi\cup\phi$ so adding an element in the image of $d$ keeps $\psi\cup\phi$ in the same equivalence class.
\end{proof}

\begin{definition}
We call the induced map $\cup:H^k(X;R)\times H^l(X;R)\to H^{k+l}(X;R)$ the \textbf{cup product} on cohomology.
\end{definition}

\begin{definition}
A \textbf{graded ring} is a ring such that its additive group is a direct sum of abelian groups $\bigoplus_{n=0}^\infty R_n$ such that $R_mR_n\subseteq R_{m+n}\forall m,n\in\mathbb{N}_0$. An element of $R_n$ is said to have \textbf{degree} $n$. A graded ring is \textbf{Graded-commutative} if we have $xy=(-1)^{|x||y|}yx$, where $|x|$ and $|y|$ denote the degrees of $x$ and $y$.
\end{definition}

\begin{proposition}
$H^*(X;R)=\bigoplus_{n=0}^\infty H^n(X;R)$ is a graded ring with multiplication given by the cup product.
\end{proposition}
\begin{proof}
The direct sum of abelian groups is abelian so $H^*(X;R)$ is abelian as an additive group.
To show that the product is associative and distributive it suffices to show that it is for cochains.\\
For associativity:
\begin{align*}
(\psi\cup\phi)\cup\theta(\sigma)&=(\psi(\sigma\circ F_{[e_0,...,e_k]})\phi(\sigma\circ F_{e_k,...,e_{k+l}}))\theta(\sigma\circ F_{[e_{k+l},...,e_{k+l+m}]})\\&=\psi(\sigma\circ F_{[e_0,...,e_k]})(\phi(\sigma\circ F_{[e_k,...,e_{k+l}]})\theta(\sigma\circ F_{[e_{k+l},...,e_{k+l+m}]}))\\&=(\psi\cup\phi)\cup\theta(\sigma).
\end{align*}
For distributivity:
\begin{align*}
\psi\cup(\phi+\theta)(\sigma)&=\psi(\sigma\circ F_{[e_0,...,e_i]})(\phi(\sigma\circ F_{[e_i,...,e_{i+j}]})+\theta(\sigma\circ F_{[e_i,...,e_{i+j}]}))\\&=\psi(\sigma\circ F_{[e_0,...,e_i]})\phi(\sigma\circ F_{[e_i,...,e_{i+j}]})+\psi(\sigma\circ F_{[e_0,...,e_i]})\theta(\sigma\circ F_{[e_i,...,e_{i+j}]})\\&=\psi\cup\phi(\sigma)+\psi\cup\theta(\sigma).
\end{align*}


\noindent If $R$ has an identity element then there is also an identity element given by $[1]\in H^0(X;R)$ which sends each singular 0-simplex to 1.
\end{proof}

\begin{proposition}
Let $R$ be a commutative ring, $\psi\in H^k(X;R)$ and $\phi\in H^l(X;R)$. Then \[\psi\cup\phi=(-1)^{kl}\phi\cup\psi.\] Thus $H^*(X;R)$ is graded-commutative.
\end{proposition}
\begin{proof}
Long and tedious. See Hatcher.
\end{proof}

\begin{proposition}
Let $f\colon X\to Y$ be a continuous map. Then $f^*(\psi\cup\phi)=f^*(\psi)\cup f^*(\phi)$.
\end{proposition}
\begin{proof}
As before, it suffices to check this for cochains. Let $\sigma\colon\Delta^{p+q}\to X$ be a singular simplex. Then 
\begin{align*}(f^*\psi\cup f^*\phi)(\sigma)&=f^*\psi(\sigma\circ F_{[e_0,...,e_p]})f^*\phi(\sigma\circ F_{[e_p,...,e_{p+q}]})\\&=\psi(f\circ\sigma\circ F_{[e_0,...,e_p]})\phi(f\circ\sigma\circ F_{[e_p,...,e_{p+q}]})\\&=(\psi\cup\phi)(f\circ\sigma)\\&=f^*(\psi\cup\phi)(\sigma).\qedhere
\end{align*}
\end{proof}

\noindent This makes $f^*$ a ring homomorphism.

\section{Cap product and Poincare duality}

\begin{definition}
Let $X$ be a topological space and $R$ a commutative ring with an identity element. The \textbf{cap product} $C_{i+j}(X;R)\times C^i(X;R)\to C_j(X;R):(c,\phi)\mapsto c\cap\phi$ given by $\sigma\cap\phi=\phi(\sigma\circ F_{[e_0,...,e_i]})\sigma\circ F_{[e_i,...,e_{i+j}]}$ for $\sigma\colon\Delta^j\to X$ a singular simplex and $\phi\in C^{i+j}(X;R)$ and extended linearly.
\end{definition}

\begin{lemma}
$\partial(\sigma\cap\phi)=(-1)^i(\partial\sigma\cap\phi-\sigma\cap d\phi)$.
\end{lemma}
\begin{proof}
We have
\[\partial\sigma\cap\phi=\sum_{t=0}^i(-1)^t\phi(\sigma\circ F_{[e_0,...,\overset{\wedge}{e_t},..,e_{i+1}]})\sigma\circ F_{[e_{i+1},...,e_{i+j}]}+\sum_{t=i+1}^{i+j}(-1)^t\phi(\sigma\circ F_{[e_0,..,e_{i}]})\sigma\circ F_{[e_{i},...,\overset{\wedge}{e_t},...,e_{i+j}]},\]
\[\sigma\cap d\phi=\sum_{t=0}^{i+1}(-1)^t\phi(\sigma\circ F_{[e_0,...,\overset{\wedge}{e_t},..,e_{i+1}]})\sigma\circ F_{[e_{i+1},...,e_{i+j}]},\]
\[\partial(\sigma\cap\phi)=\sum_{t=i}^{i+j}(-1)^{t-i}\phi(\sigma\circ F_{[e_0,..,e_{i}]})\sigma\circ F_{[e_{i},...,\overset{\wedge}{e_t},...,e_{i+j}]}.\] Thus \begin{align*}\text{RHS}&=(-1)^i((-1)^{i+1+1}\phi(\sigma\circ F_{[e_0,...,e_i]})\sigma\circ F_{[e_{i+1},...,e_{i+j}]}+\sum_{t=i+1}^{i+j}(-1)^t\phi(\sigma\circ F_{[e_0,..,e_{i}]})\sigma\circ F_{[e_{i},...,\overset{\wedge}{e_t},...,e_{i+j}]})\\&=(-1)^i(\sum_{t=i}^{i+j}(-1)^t\phi(\sigma\circ F_{[e_0,..,e_{i}]})\sigma\circ F_{[e_{i},...,\overset{\wedge}{e_t},...,e_{i+j}]})\\&=\sum_{t=i}^{i+j}(-1)^{t-i}\phi(\sigma\circ F_{[e_0,..,e_{i}]})\sigma\circ F_{[e_{i},...,\overset{\wedge}{e_t},...,e_{i+j}]}=\partial(\sigma\cap\phi).\qedhere\end{align*}
\end{proof}

\begin{proposition}
There is an induced cap product $H_{i+j}(X;R)\times H^i(X;R)\to H_j(X;R)$ which is $R$-linear in each variable.
\end{proposition}
\begin{proof}
The cap product of a cycle and a cocycle is a cycle. $\partial\sigma=0\implies\partial(\sigma\cap\phi)=\pm(\sigma\cap d\phi)$ so the cap product of a cycle and a coboundary is a boundary. $d\phi=0\implies\partial(\sigma\cap\phi)=\pm(\partial\sigma\cap\phi)$ so the cap product of a boundary and a cocycle is a boundary.
\end{proof}

\begin{proposition}
Let $f\colon X\to Y$ be continuous, $c\in C_{i+j}(X;R)$ and $\phi\in C^i(X;R)$. Then \[f_*(c)\cap\phi=f_*(c\cap f^*(\phi)).\]
\end{proposition}
\begin{proof}
Let $\sigma\colon\Delta^{i+j}\to X$ be a singular simplex. Then $f\sigma\cap\phi=\phi(f\sigma\circ F_{[e_0,...,e_j]})f\sigma\circ F_{[e_j,...,e_{i+j}]}$.
The result follows by linearity.
\end{proof}

\begin{definition}
A topological space $M$ is an \textbf{n-dimensional manifold} if for every $p\in M$ there is an open neighbourhood $U\subseteq M$ containing $p$ and a homeomorphism $f\colon U\to V$ where $V\subseteq\mathbb{R}^n$ is open.
\end{definition}

\begin{definition}
A manifold is \textbf{closed} if it is compact and has no boundary.
\end{definition}

\begin{definition}
An n-dimensional closed manifold $M$ is \textbf{orientable} if $H_n(M)\cong \mathbb{Z}$.
\end{definition}

\begin{definition}
Let $M$ be a closed n-dimensional manifold and $R$ a commutative ring with an identity. $M$ is \textbf{R-orientable} if $H_n(M;R)\cong R$. The image of a unit in $R$ is called a \textbf{fundamental class}. An \textbf{orientation} of $M$ is a choice of fundamental class $[M]\in H_n(M)$.
\end{definition}

\begin{proposition}
Let $M$ be a closed n-dimensional manifold and $R$ a commutative ring with an identity. Then $H_n(M;R)\cong R$ if $M$ is orientable and $T_2R$ otherwise. Furthermore, $H_i(M;R)=0$ for $i>n$.
\end{proposition}

\begin{definition}
Let $M$ and $N$ be closed connected orientable $n$-dimensional manifolds, let $f\colon M\to N$ be continuous and let $[M]$ and $[N]$ be fundamental classes of $X$ and $Y$. Then the \textbf{degree} of $f$ is the integer $\text{deg}(f)$ such that $f_*([M])=\text{deg}(f)[N]$.
\end{definition}

\begin{proposition}
Let $M$ and $N$ be closed R-orientable n-dimensional manifolds and $f\colon M\to N$ a homeomorphism. Then $f_*$ sends a fundamental class of $M$ to a fundamental class of $N$.
\end{proposition}
\begin{proof}
$f_*$ is an isomorphism, and isomorphisms map units to units.
\end{proof}

\begin{corollary}
The degree of a homeomorphism between closed connected orientable $n$-manifolds is $\pm 1$.
\end{corollary}

\begin{definition}
Let $M$ be an orientable n-dimensional manifold, let $[M]\in H_n(X)$ be a fundamental class and let $f\colon M\to M$ be a homeomorphism. We say that $f$ is \textbf{orientation-preserving} if $f_*([M])=[M]$ and that $f$ is \textbf{orientation-reversing} if $f_*([M])=-[M]$.
\end{definition}




\begin{theorem}
(Poincare Duality). If $M$ is a closed $R$-orientable $n$-manifold with fundamental class $[M]\in H_n(M;R)$, then the map $D\colon H^k(M;R)\to H_{n-k}(M;R)$ defined by $D([\alpha])=[M]\cap[\alpha]$ is an isomorphism for all $k$.
\end{theorem}

\begin{proposition}
Let $\psi\in C^l(X;R)$, $\alpha\in C_{k+l}(X;R)$ and $\psi\in C^k(X;R)$. Then \[\psi(\alpha\cap\phi)=(\phi\cap\psi)(\alpha).\]
\end{proposition}
\begin{proof}
Let $\sigma\colon\Delta^{k+l}\to X$ be a singular simplex. Then \[\psi(\sigma\cap\phi)=\psi(\phi(\sigma\circ F_{[e_0,...,e_k]})\sigma\circ F_{[e_k,...,e_{k+l}]})=\phi(\sigma\circ F_{[e_0,...,e_k]})\psi(\sigma\circ F_{[e_k,...,e_{k+l}]}).\]
The result then follows by linearity.
\end{proof}

\section{Cell complexes}
\begin{definition}
Let $Y$ be a topological space and let $\phi\colon S^{n-1}\to Y$ be a continuous map. Define a relation on $Y\coprod B^n$ generated by $x\sim y$ if $x\in S^{n-1}\cong\partial B^n$ and $y=\phi(x)$ and let $Y\cup_\phi B^n=Y\coprod B^n/\sim$ with the quotient topology. We call this $Y$ with an n-cell attached via $\phi$.
\noindent As a set, $Y\cup_\phi B^n=Y\coprod e^n$ where the \textbf{n-cell} $e^n$ is the image of the interior of $B^n$.
\end{definition}

\begin{definition}
A \textbf{finite cell complex} $X$ of dimension 0 is a discrete finite set $X^0$ of $0$-cells. A finite cell complex $X$ of dimension n is the result of attaching a finite number of $n$-cells to a finite cell complex $X^{n-1}$ of dimension $n-1$ using attaching maps $\phi\colon S^{n-1}\to X^{n-1}$.

\noindent The subset $X^k\subset X$ is called the \textbf{k-skeleton}.
\end{definition}

\noindent So a cell complex is formed by inductively gluing n-disks along their boundaries to a cell complex of dimension $n-1$.

\begin{definition}
Let $X^n$ be a finite cell complex. Then $CC_k(X)$ is the free abelian group generated by the $k$-cells $\{e^k_1,...,e^k_{r_k}\}$, so that $CC_k(X)\cong\mathbb{Z}^{r_k}$. \[CC_k(X;G)=CC_k(X)\otimes_\mathbb{Z}G\] and \[CC^k(X;G)=\text{Hom}(CC_k(X),G)\]for an abelian group $G$.
\end{definition}

\begin{definition}
Let $e_i^k$ be a $k$-cell with an attaching map $\phi_i\colon S^{k-1}\to X^{k-1}$. Given a $(k-1)$-cell $e_j^{k-1}$ we have a map $\phi_i^j:\phi_i^{-1}(e_j^{k-1})\to e_j^{k-1}:x\mapsto\phi_i(x)$. Then We define the boundary map $\partial\colon CC_k(X)\to CC_{k-1(X)}$ by $\partial e_i^k=\sum_jdeg(\phi_i^j)e_j^{k-1}$ and extending linearly.
\end{definition}

\begin{definition}
$d\colon CC^k(X;G)\to CC^{k+1}(X;G)$ is the dual of $\partial$, defined the same way as for singular homology.
\end{definition}

\begin{lemma}
$\partial\circ\partial=0$ and $d\circ d=0$.
\end{lemma}

\begin{definition}
The resultant homology and cohomology groups $H_k(CC_k(X;G),\partial)$ and $H^k(CC^k(X;G),d)$ are called the \textbf{cellular homology} and \textbf{cellular cohomology} groups respectively.
\end{definition}

\begin{proposition}
Let $G$ be an abelian group. Then\[H_k(CC_k(X;G),\partial)\cong H_k(X;G)\] and \[H^k(CC^k(X;G),d)\cong H^k(X;G)\] for all $k\in\mathbb{N}$.
\end{proposition}

\section{Delta complexes}
\begin{definition}
A $\Delta$--complex structure on a space $X$ is a collection of maps $\sigma_\alpha\colon\Delta^n\to X$, with $n$ depending on the index $\alpha$, such that:
\begin{enumerate}
    \item The restriction of $\sigma_\alpha$ to the interior of $\Delta^n$ is injective, and each point of $X$ is in the image of exactly one such restriction.
    \item Each restriction of $\sigma_\alpha$ to a face of $\Delta^n$ is one of the maps $\sigma_\beta\colon\Delta^{n-1}\to X$.
    \item A set $A\subset X$ is open iff $\sigma_\alpha^{-1}$ is open in $\Delta^n$ for each $\sigma_\alpha$.
\end{enumerate}
\end{definition}

The homology and cohomology groups arising from a delta-complex structure of a topological space are isomorphic to those arising from a singular structure. The chain complexes of delta-complexes are usually simpler than those in the singular case. This will be seen in the computation of the torsion linking form of Lens spaces, since finite bases for the chain groups can be found, thereby making it easy to characterize cochains by their actions on said basis.

\section{Universal coefficient theorem}
\begin{proposition}
If $G$ is a free abelian group or $G\cong\mathbb{Q}$, then $H_k(X)\otimes_\mathbb{Z}G\cong H_k(X;G)$.\\
If $G=\mathbb{Z}_p$, then \[\frac{H_k(X;\mathbb{Z}_p)}{H_k(X)\otimes_\mathbb{Z}\mathbb{Z}_p}\cong T_p(H_{k-1}(X)).\]
\end{proposition}

\begin{proposition}
Let $R$ be a commutative ring and let $A$ be an $R$-module.There is a natural inclusion $H_*(X;R)\otimes_RA\xhookrightarrow{}H_*(X;A)$\\
which is an isomorphism if $A$ is free.
\end{proposition}

In particular, $H_*(X;\mathbb{R})\cong H_*(X;\mathbb{Q})\otimes_\mathbb{Q}\mathbb{R}\cong H_*(X)\otimes_\mathbb{Z}\mathbb{R}$. If $H_k(X)$ is finitely generated, then \[\text{rk }H_k(X)=\text{dim }H_k(X;\mathbb{R}).\]

Let $G$ be an abelian group. Recall the natural pairing $H^k(X;G)\times H_k(X)\to G$ given by \[([\psi],[c])\mapsto\psi(c).\]

\begin{proposition}
The induced homomorphism \[H^k(X;G)\to\text{Hom}(H_k(X);G)\]is surjective.
\end{proposition}
Now consider the case $G=\mathbb{Z}$ and set\[K=\text{ker}(H^k(X;\mathbb{Z})\to\text{Hom}(H_k(X),\mathbb{Z}))\]
\begin{lemma}
Let $[\psi]\in K$ and $[c]\in H_{k-1}(X)$ such that $n[c]=0$ for some $n\in\mathbb{N}$. Pick $b\in C_k(X)$ such that $nc=\partial b$. Then $\psi(b)\text{ mod }n$ depends only on $[\psi]$ and $[c]$.
In particular, there is a well-defined pairing $K\times T(H_{k-1}(X))\to\mathbb{Q}/\mathbb{Z}$ given by $([\psi],[c])\mapsto\frac{1}{n}\psi(b)$.
\end{lemma}

\begin{proposition}
If $H_*(X)$ is finitely generated, then $K\to\text{Hom}(T(H_{k-1}(X)),\mathbb{Q}/\mathbb{Z})$ is an isomorphism.
\end{proposition}

\begin{remark}
An isomorphism can also be induced by $([\psi],[c])\mapsto\frac{1}{k}\gamma(c)$, where $\gamma\in C^{k-1}(X;\mathbb{Z})$ such that $d\gamma=k\phi$ for some $k\in\mathbb{N}$.
\end{remark}

\begin{corollary}
$K=T(H^k(X;\mathbb{Z}))$.
\end{corollary}
\begin{corollary}
If $H_*(X)$ is finitely generated, then the free part of $H^k(X;\mathbb{Z})$ is naturally dual to the free part of $H_k(X;\mathbb{Z})$ and the torsion part of $H^k(X;\mathbb{Z})$ is naturally dual to the torsion part of $H_{k-1}(X)$.
\end{corollary}

\begin{proposition}
If $A$ is an $R$-module, then the natural homomorphism\[H^k(X;A)\to\text{Hom}_R(H_k(X;R),A)\]is surjective. If $R$ is a field, then it is an isomorphism.\\
So for a field $\mathbb{F}$, we have $H_k(X;\mathbb{F})\cong H_k(X;\mathbb{F})^*$.
\end{proposition}

\section{Proof of the Borsuk-Ulam theorem with cohomology}

\textrm{(Hatcher section 3.2 exercise 3) \\}
We begin with the following lemmas.

\begin{lemma}
There is no map $\mathbb{R} P^n$$\rightarrow$$\mathbb{R} P^m$ inducing a nontrivial map $H^1(\mathbb{R} P^m;\mathbb{Z}_2)\rightarrow H^1(\mathbb{R} P^n;\mathbb{Z}_2)$ if $n>m$.
\end{lemma}
\begin{proof}
Let $x_n$ and $x_m$ be the generators of $H^1(\mathbb{R} P^n;\mathbb{Z}_2)$ and $H^1(\mathbb{R} P^m;\mathbb{Z}_2)$ respectively.
We have
\[H^1(\mathbb{R} P^n;\mathbb{Z}_2)\cong \mathbb{Z}_2\] and\[H^*(\mathbb{R} P^n;\mathbb{Z}_2)\cong \mathbb{Z}_2[x]/\langle x^{n+1}\rangle.\] 

\noindent Let $f\colon\mathbb{R} P^n\to\mathbb{R} P^m $ be such a map so that $f^*(x_m)=x_n$.
Then since $f^*$ is a homomorphism on the cup product structure, $f^*(0)=f^*(x_m^{m+1})=f^*(x_m)^{m+1}=x_n^{m+1}=0$, requiring $m \geq n$. The result follows by contraposition.
\end{proof}

\begin{lemma}
If $X$ is a path-connected topological space then $H_1(X;\mathbb{Z}_2)$ is the two-elementary part of $\pi_1(X)^{\text{ab}}$, i.e. the quotient by the subgroup generated by all squares.
\end{lemma}
\begin{proof}
We have that $H_0(X)\cong\mathbb{Z}$ so by the universal coefficient theorem \[\frac{H_1(X;\mathbb{Z}_2)}{H_1(X)\bigotimes_\mathbb{Z}\mathbb{Z}_2}\cong T_2(H_0(X))=0\] The correspondence theorem then gives $H_1(X;\mathbb{Z}_2)=H_1(X)\bigotimes_\mathbb{Z}\mathbb{Z}_2\cong\pi_1(X)^{\text{ab}}\bigotimes_\mathbb{Z}\mathbb{Z}_2$.
\end{proof}

\begin{lemma}
Given any continuous map $f\colon\mathbb{R}P^n\to \mathbb{R}P^m$ with $n>m\geq 0$, $f_*\colon\pi_1(\mathbb{R}P^n)\to \pi_1(\mathbb{R}P^m)$ is trivial.
\end{lemma}
\begin{proof}
 $f^*\colon H^1(\mathbb{R}P^m;\mathbb{Z}_2)\rightarrow H^1(\mathbb{R}P^n;\mathbb{Z}_2)$ is dual to $f_*\colon H_1(\mathbb{R}P^n;\mathbb{Z}_2)\rightarrow H_1(\mathbb{R}P^m;\mathbb{Z}_2)$ which is equal to $f_*\colon \pi_1(\mathbb{R}P^n)\rightarrow \pi_1(\mathbb{R}P^m)$ by the Hurewicz theorem and the fact that $\pi_1(\mathbb{R}P^k)$ is abelian. Thus $f_*$ is trivial.
\end{proof}

\noindent This then entails a proof of the Borsuk-Ulam theorem by applying the ultimate lifting theorem to derive a contradiction.


\section{Lens spaces}
\begin{definition}
The \textbf{3-sphere } $S^3=\{(u,v)\in\mathbb{C}^2:|u|^2+|v|^2=1\}$.
\end{definition}

\begin{definition}
Let $p$ and $q\in\mathbb{N}$ be coprime. Then the \textbf{lens space} $L(p;q)$ is the quotient of $S^3$ by $\mathbb{Z}_p$ with the action generated by $(v,w)\mapsto(e^{2\pi i/p}v,e^{2\pi iq/p}w)$.
\end{definition}

\begin{example}
$L(1;q)\cong S^3$.
\end{example}
\begin{proof}
The group action is generated by $(v,w)\mapsto(e^{2\pi i}v,e^{2\pi iq}w)=(v,w)$ and so is the trivial action. This gives a homeomorphism $S^3\to L(1;q):(v,w)\mapsto[(v,w)]$.
\end{proof}

\begin{example}
$L(2;1)\cong\mathbb{R}P^3$.
\end{example}
\begin{proof}
The group action is generated by $(v,w)\mapsto(e^{\pi i}v,e^{\pi i}w)$ and so $(v,w)\sim(-v,-w)=-(v,w)$ which is the identification which gives $\mathbb{R}P^3$.
\end{proof}

\noindent Thus lens spaces can be considered generalisations of real projective space.

\begin{proposition}
$H_1(L(p;q))\cong\pi_1(L(p;q))\cong \mathbb{Z}_p$.
\end{proposition}
\begin{proof}
The multiplication map induced by the group action is clearly continuous for all elements of $\mathbb{Z}_p$.
Now let $(a,b)\in S^3$. If $S=(B_{|a|\text{sin}(\frac{\pi}{100p})}(a)\cap |a|S^1,B_{|b|\text{sin}(\frac{\pi}{100p})}(b)\cap |b|S^1)$ then $gS\cap S=\emptyset$ for all non-trivial $g\in \mathbb{Z}_p$ so the action is a covering space action giving $\pi_1(L(p;q))\cong \mathbb{Z}_p$. $H_1(L(p;q))\cong \mathbb{Z}_p$ by the Hurewicz theorem.
\end{proof}

\noindent Then $H^2(L(p;q);\mathbb{Z})\cong\mathbb{Z}_p$ by Poincare duality. $H_3((L(p;q))\cong\mathbb{Z}\cong H^0(L(p;q);\mathbb{Z})$ since $L(p;q)$ is an orientable 3-manifold. $H_0(L(p;q))\cong\mathbb{Z}\cong H^3(L(p;q);\mathbb{Z})$ since $L(p;q)$ is path-connected. $T(H_2(L(p;q)))\cong T(H^3(L(p;q));\mathbb{Z})=0$ and $F(H_2(L(p;q)))\cong F(H^2(L(p;q);\mathbb{Z}))=0$ by the universal coefficient theorem so $H_2(L(p;q))=0=H^1(L(p;q);\mathbb{Z})$. $H_i(L(p;q))=H^i(L(p;q);\mathbb{Z})=0\forall i>3$ since $L(p;q)$ is a 3-manifold. The homology and cohomology groups depend only on $p$ and are insufficient for a homotopy-equivalence classification.

\begin{definition}
Let $M^n$ be a closed oriented manifold with fundamental class $[M]\in H_n(M;\mathbb{Z})$. Let $[\alpha]\in H^a(M;\mathbb{Z})$ and $[\beta]\in H^{n-a+1}(M;\mathbb{Z})$ be torsion classes, i.e. $k[\alpha]=0,m[\beta]=0$ for some $k,m\in\mathbb{N}$. Then we can write $k\alpha=d\gamma$ for some $\gamma\in C^{a-1}(M,\mathbb{Z})$. We define $[\alpha]\vee[\beta]=\frac{1}{k}(\gamma\cup\beta)M\in\mathbb{Q}/\mathbb{Z}$. The \textbf{torsion linking pairing} is then \[T(H^a(M;\mathbb{Z}))\times T(H^{n-a+1}(M;\mathbb{Z}))\to\mathbb{Q}/\mathbb{Z}:([\alpha],[\beta])\mapsto[\alpha]\vee[\beta].\]
\end{definition}
\begin{lemma}
$[\alpha]\vee[\beta]$ is well-defined.
\end{lemma}
\begin{proof}
Let $d\gamma=d\eta=k\alpha$ for $\eta\in C^{a-1}(M,\mathbb{Z})$. Then $d(\gamma-\eta)=0$ so $\gamma-\eta$ is a representative of an element of $H^{a-1}(M;\mathbb{Z})$. \[((\gamma-\eta)\cup\beta)M=([\gamma-\eta]\cup[\beta])M\] so \[m((\gamma-\eta)\cup\beta)M=([\gamma-\eta]\cup[m\beta])M=0.\] Thus $m(\gamma\cup\beta)M=m(\eta\cup\beta)M$ so $(\gamma\cup\beta)M=(\eta\cup\beta)M$, implying independence of $\gamma$.

\noindent Now let $[\alpha]=[\alpha']\in H^a(M;\mathbb{Z})$. Then $\alpha'=\alpha+de$ for some $e\in C^{a-1}(M,\mathbb{Z})$, so \[k\alpha'=d\gamma+dke=d(\gamma+ke).\] \[\frac{1}{k}((\gamma+ke)\cup\beta)M=\frac{1}{k}(\gamma\cup\beta)M+\frac{1}{k}k(e\cup\beta)M=\frac{1}{k}(\gamma\cup\beta)M \in \mathbb{Q}/\mathbb{Z}\] implying independence of $\alpha$.

\noindent Finally, let $[\beta]=[\beta']\in H^{n-a+1}(M;\mathbb{Z})$. Then $\beta'=\beta+df$ for some $f\in C^{n-a}(M,\mathbb{Z})$. \[d(\gamma\cup f)=k(\alpha\cup f)+(-1)^{a-1}(\gamma\cup df)\] so \[\frac{1}{k}(\gamma\cup df)M=\frac{(-1)^{1-a}}{k}d(\gamma\cup f)M=\frac{(-1)^{1-a}}{k}([0],[M])=0.\] Thus \[\frac{1}{k}(\gamma\cup\beta')M=\frac{1}{k}(\gamma\cup(\beta+df))M=\frac{1}{k}(\gamma\cup\beta)M+\frac{1}{k}(\gamma\cup df)M=\frac{1}{k}(\gamma\cup\beta)M,\] implying independence of $\beta$. Thus the pairing is well-defined.
\end{proof}

\begin{lemma}
$[\beta]\vee[\alpha]=(-1)^{a(n-a+1)}[\alpha]\vee[\beta]$.
\end{lemma}
\begin{proof}
let $m\beta=d\tau$ for some $\tau\in C^{n-a}(M,\mathbb{Z})$. Then 
\begin{align*}
[\beta]\vee[\alpha]&=\frac{1}{m}(\tau\cup\alpha)M\\&=(-1)^{a(n-a)}\frac{1}{m}(\alpha\cup\tau)M\\&=\frac{(-1)^{a(n-a)}}{km}(k\alpha\cup\tau)M\\&=\frac{(-1)^{a(n-a)}}{km}(d\gamma\cup\tau)M.\end{align*} \[d(\gamma\cup\tau)=(d\gamma\cup\tau)+(-1)^{a-1}(\gamma\cup d\tau)=(d\gamma\cup\tau)+(-1)^{a-1}(\gamma\cup m\beta)\] so \begin{align*}[\beta]\vee[\alpha]=\frac{(-1)^{a(n-a)}}{km}(d\gamma\cup\tau)M&=\frac{(-1)^{a(n-a)}}{km}(d(\gamma\cup\tau)M-(-1)^{a-1}(\gamma\cup m\beta)M)\\&=\frac{(-1)^{a(n-a)}}{km}(-1)^a(\gamma\cup m\beta)M\\&=\frac{(-1)^{a(n-a+1)}}{k}(\gamma\cup\beta)M\\&=(-1)^{a(n-a+1)}[\alpha]\vee[\beta]\qedhere\end{align*}
\end{proof}

\begin{lemma}
The torsion linking pairing is a non-degenerate bilinear form, where non-degeneracy means that $T(H^a(M,\mathbb{Z}))\to \text{Hom}(T(H^{n-a+1}(M;\mathbb{Z})),\mathbb{Q}/\mathbb{Z}):[\alpha]\mapsto ([\beta]\mapsto [\alpha]\vee[\beta])$ is an isomorphism.
\end{lemma}
\begin{proof}
Bilinearity follows from the bilinearity of the cup product.

\noindent By the universal coefficient theorem, $T(H^{a}(M;\mathbb{Z}))\to\text{Hom}(T(H_{a-1}(M;\mathbb{Z})),\mathbb{Q}/\mathbb{Z}):[\alpha]\mapsto\gamma_{[\alpha]}$ is an isomorphism where $\gamma_{[\alpha]}([l])=\frac{1}{k}\gamma(l)$. By Poincare duality we have an isomorphism \[D\colon T(H^{n-a+1}(M;\mathbb{Z}))\to T(H_{a-1}(M;\mathbb{Z})):[\beta]\mapsto[\beta]\cap[M]\] and so \[T(H^a(M,\mathbb{Z}))\to \text{Hom}(T(H^{n-a+1}(M;\mathbb{Z})),\mathbb{Q}/\mathbb{Z}):[\alpha]\mapsto\gamma_{[\alpha]}\circ D\] is also an isomorphism where \[\gamma_{[\alpha]}\circ D([\beta])=\frac{1}{k}\gamma(\beta\cap M)=\frac{1}{k}(\gamma\cup\beta)M=[\alpha]\vee[\beta].\] Thus \[T(H^a(M,\mathbb{Z}))\to \text{Hom}(T(H^{n-a+1}(M;\mathbb{Z})),\mathbb{Q}/\mathbb{Z}):[\alpha]\mapsto ([\beta]\mapsto [\alpha]\vee[\beta])\] is an isomorphism so the bilinear form is non-degenerate.
\end{proof}
\noindent So in the case of 3-manifolds like lens spaces, the torsion linking pairing is a symmetric non-degenerate bilinear form on $T(H^2(M;\mathbb{Z}))$. Also, since the cup product is invariant under homotopy-equivalence, the torsion linking pairing is as well.


To compute the torsion linking form of a lens space, we first need to give a lens space a $\Delta$--complex structure. To begin with, we need an easily-visualised model of $S^3$.

$S^3$ is the set of pairs of complex numbers $(z_0,z_1)$ such that their norms sum to $1$. Representing them in polar coordinates then gives $z_0=r_0e^{i\theta_0}$ and $z_1=r_1e^{i\theta_1}$, with $r_0^2+r_1^2=1$. $r_1$ determines $r_0=\sqrt{1-r_1^2}$ so $(z_0,z_1)$ can be associated with a real triple $(r_1,\theta_0,\theta_1)$, where $0\leq r_1\leq 1$ and $0\leq\theta_0,\theta_1<2\pi$.
For there to be a bijection between pairs $(z_0,z_1)$ and triples $(r_1,\theta_0,theta_1)$, we require the following identifications:
\begin{enumerate}
\item $(0,\theta_0,\theta_1)\sim(0,\theta_0,\theta_1^{'})$ for all $0\leq\theta_1,\theta_1^{'}<2\pi$, since $r_1=0\implies (z_0,z_1)=(e^{i\theta_0},0)$ is independent of $\theta_1$.
\item $(1,\theta_0,\theta_1)=(1,\theta_0^{'}m,\theta_1)$ for all $0\leq\theta_0,\theta_0^{'}<2\pi$, since $r_1=0\implies (z_0,z_1)=(0,e^{i\theta_1})$ is independent of $\theta_0$.
\end{enumerate}

If we represent points of a solid torus $T$ with meridianal radius 1, where $\theta_0$ gives the longitudinal position and $\theta_1,r_1$ give the position of the point within the cross-section determined by $\theta_0$, then condition $1$ is satisfied, since all angles are identified on a circle with no radius.

The case where $r_1=1$ corresponds to the boundary of $T$. Varying $\theta_0$ then traces out a latitude, so condition $2$ corresponds to identifying each latitude of $\partial T$ with a single point.

T is homeomorphic to a solid cylinder with the ends identified, and identifying each vertical line on the boundary then gives $S^3$ due to satisfying condition $2$. Collapsing the vertical lines to points midway along the cylinder then shows that $S^3$ is homeomorphic to a solid ball with upper and lower hemisphered identified via orthogonal projection, where the lines collapsed to points on the equator.



\noindent Consider a filled-in ball with points N and S corresponding to the north and south poles and points $x_0,...,x_{p-1}$ being evenly-spaced points along the equator. Then draw a line through the ball from N to S, lines connecting $x_i$ to $x_{i+1}$, lines connecting $x_i$ to N and lines connecting $x_i$ to S (all geodesic). We then identify the faces on the surface by twisting the top hemisphere by $\frac{2\pi q}{p}$ radians and projecting downwards. This leaves us, after identification, with:\\
Points: $[N]$ and $[x_0]$ (since $\{x_0,x_q,x_{2q},...\}$ are identified and $q$ is coprime to $p$)\\
Lines: $[NS]$, $[x_0x_1]$ and $[Nx_0],...,[Nx_{p-1}]$\\
Faces: $[Nx_0x_1],...,[Nx_{p-2}x_{p-1}],[Nx_0x_{p-1}]$ and $[NSx_0],...,[NSx_{p-1}]$\\
Slices: $[NSx_0x_1],...,[NSx_{p-2}x_{p-1}],[NSx_0x_{p-1}]$\\
\[\text{Let }a_{p-1}=-[NSx_0x_{p-1}]\text{ and }a_i=[NSx_ix_{i+1}]\text{ otherwise.}\]
\[\text{Let }b_{p-1}=-[Nx_0x_{p-1}]\text{ and }b_i=[Nx_ix_{i+1}]\text{ otherwise.}\]
\[\text{Let }c_i=[NSx_i].\]
\[\text{Let }D=[x_0x_1].\]
\[\text{Let }E=[NS].\]
\[\text{Let }F_i=[Nx_i].\]
\[-\partial b_{p-1}=\partial[Nx_0x_{p-1}]=[x_0x_{p-1}]-[Nx_{p-1}]+[Nx_0]=[Nx_0]-[Nx_{p-1}]-[x_0x_1]\]
Otherwise, \[\partial b_i=\partial[Nx_ix_{i+1}]=[x_0x_1]-[Nx_{i+1}]+[Nx_i]\]
so \[\partial b_i=D-F_{i+1}+F_i.\]
\[\partial c_i=\partial[NSx_i]=[Sx_i]-[Nx_i]+[NS]=[NS]-[Nx_i]+[Nx_{i-q}]\]
so \[\partial c_i=E-F_i+F_{i-q}.\]
\[\partial(\sum_{i=0}^{p-1}a_i)=\sum_i([Nx_{i-q}x_{i-q+1}]-[Nx_ix_{i+1}]+[NSx_{i+1}]-[NSx_i])=0\] so $\sum_{i=0}^{p-1}a_i$ is a representative of an element of $H_3(L(p;q);\mathbb{Z})$. In fact, it is a representative of a fundamental class. Call the equivalence class it represents $[M]$.\\
$C_1(L(p;q),\mathbb{Z})$ has basis $D,E,F_i$ and $C_2(L(p;q),\mathbb{Z})$ has basis $b_i,c_i$.\\
Let $\hat X$ be linear map which sends $X$ to $1$ and all other elements of the basis to $0$. We then have:\\
\[d\hat D=\sum_{i=0}^{p-1}\hat b_i.\]
\[d\hat E=\sum_{i=0}^{p-1}\hat c_i.\]
\[d\hat F_i=\hat b_i-\hat b_{i-1}-\hat c_i+\hat c_{i+q}.\]
Then define \[\gamma=\hat D + (\sum_{i=0}^{p-1}i\hat F_i)+q\hat E\in C^1(L(p;q);\mathbb{Z}).\]
\begin{align*}
d\gamma&=\sum_{i=0}^{p-1}\hat b_i+(\sum_{i=0}^{p-1}i(\hat b_i-\hat b_{i-1}-\hat c_i+\hat c_{i+q}))+\sum_{i=0}^{p-1}q\hat c_i\\
&=\sum_{i=0}^{p-1}((i+1)\hat b_i-i\hat b_{i-1}-(i-q)\hat c_i+i\hat c_{i+q})\equiv 0\text{ mod }p.\\
\end{align*}
Let $\alpha=\frac{1}{p}d\gamma\in C^2(L(p;q);\mathbb{Z})$. Then $\alpha$ is a representative of an element in $H^2(L(p;q);\mathbb{Z})$ with $p[\alpha]=[0]$. Also note that $\gamma(E)=q$.
\[p\alpha([Sx_{q-1}x_q])=p\alpha([Nx_{p-1}x_0])=p\alpha(b_{p-1})=(p-1+1)\hat b_{p-1}(b_{p-1})-0\times\hat b_{-1}(b_{p-1})=p\]\[\implies \alpha([Sx_{q-1}x_q])=1.\]
Otherwise, \[p\alpha([Sx_ix_{i+1}])=p\alpha([Nx_{i-q}x_{i+1-q}])=p\alpha(b_{i-q})=(kp+i-q+1)\hat b_{{i-q}}(b_{{i-q}})-(kp+i-q+1)\hat b_{{i-q}}(b_{{i-q}})=0\]where $k\in\mathbb{Z}$ puts $kp+i-q+1$ in the compatible range for the sum of $[1,p-1]$ (The fact that it can't wrap around to $0$ is why cancellation occurs).\\
Thus, \[[\alpha]\vee[\alpha]=\frac{1}{p}\gamma([NS])\alpha([Sx_{q-1}x_q])=\frac{q}{p}.\]

\noindent We then have that if $L(p;q_1)$ and $L(p;q_2)$ are homotopy-equivalent, then $\frac{q_1}{p}=\pm\frac{q_2}{p}m^2$ for some integer $m$, taking into account whether or not the homotopy-equivalence is orientation-reversing and choice of element of $H^2(L(p;q);\mathbb{Z})$. This is equivalent to $q_1\equiv\pm q_2m^2\text{ mod }p\iff q_1q_2^{-1}\equiv\pm m^2\text{ mod }p\iff q_1q_2\equiv\pm(mq_2)^2\text{ mod } p\iff q_2q_2\equiv\pm n^2\text{ mod } p$, where $n=mq_2$.

If we were only interested in orientation-preserving homotopy-equivalence, then the condition would be $q_2q_2\equiv n^2\text{ mod } p$; and if we were only interested in orientation-reversing homotopy-equivalence, then the condition would be $q_2q_2\equiv- n^2\text{ mod } p$.
Thus $L(p;q)$ is orientation-reversing homotopy-equivalent to itself iff $q(-q)$ is a square mod $p$ $\iff -1$ is a square mod $p\iff p\equiv 1$ mod $4$ for $p$ a prime.

\begin{definition}
Let $X$ be a topological space and $G$ a group acting on it. A \textbf{fundamental domain} is a subset of $X$ which contains precisely one point from each orbit.
\end{definition}

\noindent As such, a fundamental domain is in bijection with $X/G$.

\noindent To construct a lens space as a cell complex we first find a fundamental domain.
\[L(p;q)=\{(r_1e^{2\pi i\theta_1},r_2e^{2\pi i\theta_2}):r_1^2+r_2^2=1\}/(\theta_1,\theta_2)\sim(\theta_1+\frac{1}{p},\theta_2+\frac{q}{p}).\]

\noindent Let $\rho\colon S^3\to S^3$ represent the action.

\noindent The action rotates the first coordinate by $1/p$ so let \[e_k^3=\{(r_1e^{2\pi i\theta_1},r_2e^{2\pi i\theta_2})\in S^3:\frac{k}{p}<\theta_1<\frac{k+1}{p}\}.\]

\noindent The $2$-cells will then be the boundaries of the $3$-cells, so let \[e_k^2=\{(r_1e^{2\pi i\theta_1},r_2e^{2\pi i\theta_2})\in S^3:\theta_1=\frac{k}{p}\}.\]

\noindent Fixing $\theta_1$ leaves $r_2e^{2\pi i\theta_2}$ moving in a circle, so let \[e_k^1=\{(r_1e^{2\pi i\theta_1},r_2e^{2\pi i\theta_2})\in S^3:\theta_1=0,\frac{k}{p}<\theta_2<\frac{k+1}{p}\}.\]

\noindent Then let $e_k^0=\{(r_1e^{2\pi i\theta_1},r_2e^{2\pi i\theta_2})\in S^3:\theta_1=0,\theta_2=\frac{k}{p}\}$ be the boundary of $e_k^1$.

\noindent Then we have $\rho(e_k^3)=e_{k+1}^3$, $\rho(e_k^2)=e_{k+1}^2$, $\rho(e_k^1)=e_{k+q}^1$ and $\rho(e_k^0)=e_{k+q}^0$.

\noindent Thus $\rho$ permutes the cells so taking the quotient gives a cell complex structure \[L(p;q)=e^0\cup e^1\cup e^2\cup e^3.\]

\[\partial e_0^3=e_1^2-e_0^2=(\rho-\text{id})(e_0^2)\]
\[\partial e_1^2=e_0^1+...+e_{p-1}^1=(\text{id}+\rho+...+\rho^{p-1})(e_0^1)\]
\[\partial e_0^1=e_l^0-e_0^0=(\rho^l-\text{id})(e_0^0)\] where $lq\cong1\text{ mod }p$ (so that $l$ corresponds to rotating by $\frac{2\pi i}{p}$).

\noindent Thus on $L(p;q)$, we have $\partial e^3=0$, $\partial e^2=pe^1$, $\partial e^1=0$ since $p=\text{id}$ modulo the group action. This then gives the same homology groups as were found earlier.

\section{Reidemeister torsion}
\begin{definition}
Let $0\to C_n\overset{\partial}{\to}C_{n-1}\overset{\partial}{\to}C_{n-2}\overset{\partial}{\to}...\overset{\partial}{\to}C_{0}\to 0$ be a chain complex with no homology and bases $c_i$ of $C_i$. We can then pick auxiliary bases $b_i\cup\overset{\sim}{b}_{i-1}$ of $C_i$ where $b_i$ is a basis for $\text{Im }\partial_{i+1}$ and $\partial\overset{\sim}{b}_{i-1}=b_{i-1}$. Then let $[b_i\cup\overset{\sim}{b}_{i-1}|c_i]$ be the determinant of the change of basis matrix. We then define The \textbf{Reidemeister torsion} \[\Delta(C_.,c_.)=\prod_{i=0}^n[b_i\cup\overset{\sim}{b}_{i-1}|c_i]^{(-1)^i}.\]
\end{definition}

\begin{lemma}
$\Delta(C_.,c_.)$ is independent of choice of auxiliary bases.
\end{lemma}


\noindent Let $X=\cup_{k=0}^n\cup_je_j^k$ be a finite cell complex and $(A,\beta)$ a representation of $\pi_1(X)$ in a ring $A$. Let $\overset{\sim}{X}$ be a universal cover. And $\overset{\sim}{e}_j^k$ be a lift of $e_j^k$. Then every other lift of $e_j^k$ is obtained by composition of $\overset{\sim}{e}_j^k$ with a deck transformation (since given $f'$ and $g'$ lifts of $f$ there exists a $\phi\in\text{Aut}(p)$ such that $\phi\circ f'(0)=g'(0)$ which implies $\phi\circ f=g$ by uniqueness. $\text{Aut}(p)=\pi_1(X)$ so we can write $\overset{\sim}{X}=\cup_{k=0}^n\cup_{g\in\pi_1(X)}\cup_jg\overset{\sim}{e}_j^k$. Thus multiplication by an element of $\pi_1(x)$ permutes the cells, thereby giving $C_.^\text{cw}(\overset{\sim}{X})$ a $\pi_1(X)$-action. $A$ also has a $\pi_1(X)$ action so we can define $C_k(X;\beta)=C_k^\text{cw}(\overset{\sim}{X})\otimes_{\mathbb{Z}\pi_1(X)} A$. We have an induced boundary map $\partial\colon C_k(X;\beta)\to C_{k-1}(X;\beta)$ which makes $C_k(X;\beta)$ a chain complex with basis $\{\overset{\sim}{e}_j^k\}\otimes\text{id}_A$.

\begin{definition}
The \textbf{Reidemeister torsion of $X$ with respect to $\beta$} to be $\Delta(X;\beta)=\Delta(C_.(X;\beta)\otimes_{\mathbb{Z}\pi_1(X)}A)$ if $C_.(X;\beta)\otimes_{\mathbb{Z}\pi_1(X)}A$ has no homology. This is a well-defined element of $A^*/\{\pm\beta(\pi_1(X))\}$.
\end{definition}

\noindent This is invariant under homeomorphism but not homotopy-equivalence.



\noindent Let $\omega$ be a $p$th root of unity and let $\beta_\omega\colon\mathbb{Z}_p\to \mathbb{C}^*$ be the representation given by $1\mapsto\omega$.
In $C_.(X;\beta_\omega)\otimes_{\mathbb{Z}\pi_1(X)}\mathbb{C}^*$ we have:\\
\[\partial(\overset{\sim}{e}_0^3\otimes\text{id})=(\omega-1)(\overset{\sim}{e}_0^2\otimes\text{id}),\]
\[\partial(\overset{\sim}{e}_0^2\otimes\text{id})=(1+\omega+\omega^2+...+\omega^{p-1})(\overset{\sim}{e}_0^1\otimes\text{id}),\]
\[\partial(\overset{\sim}{e}_0^1\otimes\text{id})=(\omega^l-1)(\overset{\sim}{e}_0^0\otimes\text{id}),\] giving a complex:
\[0\to\mathbb{C}^*\overset{\omega-1}{\to}\mathbb{C}^*\overset{0}{\to}\mathbb{C}^*\overset{\omega^l-1}{\to}\mathbb{C}^*\to0\text{ whenever }\omega\neq 1.\]
These maps are isomorphisms and so the complex has no homology. Hence for $\omega\neq 1$ we have \[\Delta(L(p;q);\beta_\omega)=(\omega^l-1)(\omega-1)\in\mathbb{C}^*/\pm\omega^k\text{ for }k\in\mathbb{Z}.\]
\[(\omega^l-1)(\omega-1)=\omega^l(1-\omega^{-l})(\omega-1)=(\omega^{-l}-1)(\omega-1)=\omega(\omega^{-l}-1)(1-\omega^{-1})=(\omega^{-l}-1)(\omega^{-1}-1)\] so \[\Delta(L(p;q);\beta_\omega)=\Delta(L(p;-q);\beta_\omega).\]
It can also be shown that \[(\omega^l-1)(\omega-1)=(\omega^q-1)(\omega-1)\] so \[\Delta(L(p;q);\beta_\omega)=\Delta(L(p;q^{-1});\beta_\omega).\]
Thus two lens spaces $L(p;q_1)$ and $L(p;q_2)$ can only be homeomorphic if $q_1\equiv \pm q_2^{\pm 1}\text{mod }p$.

\begin{lemma}
Let X be a topological space. Let $G_1$ and $G_2$ be conjugate subgroups of Homeo$(X)$ defining covering space actions of $X$. Then $X/G_1\cong X/G_2$.
\end{lemma}
\begin{proof}
We have that $G_2=hG_1h^{-1}$ for some $h\in\text{Homeo}(X)$. The map $f:X\to X/G_2\colon x\mapsto[h\cdot y]$ is continuous, since $h$ acts by a homeomorphism and the quotient map is continuous. We also have that if $x'=g_1\cdot y$ for some $g_1\in G_1$, then \[f(x')=[hg_1\cdot x]=[(hg_1^{-1}h^{-1})\cdot hg_1\cdot x]=[h\cdot x]=f(x).\] The universal property of quotient maps then implies that there exists a continuous map $\bar f:X/G_1\to X/G_2$. Similarly, there exists a continuous map $\bar g:X/G_2\to X/G_1\colon [x]\mapsto[h^{-1}\cdot x]$. \[\bar g\circ\bar f([x])=[h^{-1}\cdot (h\cdot x)]=[x].\] Similarly, $\bar f\circ\bar g([x])=[x]$. Thus $X/G_1\cong X/G_2$.
\end{proof}

\begin{lemma}
$L(p;q)\cong L(p;-q)$
\end{lemma}
\begin{proof}
$(v,w)\mapsto(e^{2\pi i/p}v,e^{2\pi iq/p}w)$ and $(v,w)\mapsto(e^{2\pi i/p}v,e^{-2\pi iq/p}w)$ are conjugate by the homeomorphism $(v,w)\mapsto(v,\bar w)$. The result follows by the above lemma.
\end{proof}
\begin{lemma}
If $q_1\equiv q_2^{-1}\text{ mod }p$ then $L(p;q_1)\cong L(p;q_2)$.
\end{lemma}
\begin{proof}
We have that $q_1q_2\equiv 1\text{ mod }p$.
\[{(e^{2i\pi/p})}^{q_1}=e^{2i\pi q_1/p},{(e^{2i\pi/p})}^{q_2}=e^{2i\pi q_2/p}\] and \[{(e^{2i\pi q_1/p})}^{q_2}=e^{2i\pi /p}={(e^{2i\pi q_2/p})}^{q_1}.\] Thus the subgroup generated by \[(v,w)\mapsto(e^{2i\pi /p}v,e^{2i\pi q_1/p}w)\] is the same as the subgroup generated by \[(v,w)\mapsto(e^{2i\pi q_2/p}v,e^{2i\pi/p}w).\] Furthermore, \[(v,w)\mapsto(e^{2i\pi q_2/p}v,e^{2i\pi/p}w)\] is conjugate to \[(v,w)\mapsto(e^{2i\pi/p}w,e^{2i\pi q_2/p}v)\] by the homeomorphism $(v,w)\mapsto(w,v)$. Thus the subgroups generated by \[(v,w)\mapsto(e^{2i\pi /p}v,e^{2i\pi q_1/p}w)\] and \[(v,w)\mapsto(e^{2i\pi /p}v,e^{2i\pi q_2/p}w)\] are conjugate so $L(p;q_1)\cong L(p;q_2)$.
\end{proof}

\begin{theorem}
Let $L(p;q_1)$ and $L(p;q_2)$ be lens spaces.\\
They are homotopy equivalent if and only if $q_1q_2\equiv\pm n^2 \text{ mod } p$ for some $n\in\mathbb{N}$.\\
They are homeomorphic if and only if $q_1\equiv\pm q_2^{\pm 1} \text{ mod }p$.
\end{theorem}

\begin{example}
$L(7;1)$ and $L(7;2)$ are homotopy-equivalent but not homeomorphic.
\end{example}
\end{document}

