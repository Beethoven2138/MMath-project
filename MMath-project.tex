\documentclass{article}
\usepackage{graphicx} % Required for inserting images
\usepackage[english]{babel}
\usepackage{amsthm}
\usepackage{amsfonts}

\title{MMath project}
\author{Saxon Supple }
\date{January 2024}

\begin{document}

\maketitle

\section{Proof of the Borsuk-Ulam theorem with cohomology}

\textrm{(Hatcher section 3.2 exercise 3) \\}
We begin by showing that there is no map
$\mathbb{R} P^n$$\rightarrow$$\mathbb{R} P^m$ inducing a nontrivial map $H^1(\mathbb{R} P^m;\mathbb{Z}_2)\rightarrow H^1(\mathbb{R} P^n;\mathbb{Z}_2)$ if $n>m$.

\begin{proof}
Let $x_n$ and $x_m$ be the generators of $H^1(\mathbb{R} P^n;\mathbb{Z}_2)$ and $H^1(\mathbb{R} P^m;\mathbb{Z}_2)$ respectively.

\[H^1(\mathbb{R} P^n;\mathbb{Z}_2)\cong \mathbb{Z}_2\] 
\[H^*(\mathbb{R} P^n;\mathbb{Z}_2)\cong \mathbb{Z}_2[x]/<x^{n+1}>\] 

Let $f:\mathbb{R} P^n\rightarrow\mathbb{R} P^m $ be such a map so that $f^*(x_m)=x_n$.
Then since $f^*$ is a homomorphism on the cup product structure, $f^*(0)=f^*(x_m^{m+1})=f^*(x_m)^{m+1}=x_n^{m+1}=0$, requiring $m \geq n$. The result follows by contraposition.
\end{proof}

\end{document}
